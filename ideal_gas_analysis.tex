\documentclass{article}
\usepackage[a4paper,left=2cm,top=2cm]{geometry}

\usepackage{parskip}
\usepackage{amsmath}
\usepackage{amssymb}
\usepackage{mathtools}
\usepackage{hyperref}

\begin{document}

\title{Analysis of Superpressure Balloon using the ideal gas law}
\author{Richard Meadows 2016}

Analysis of Superpressure Balloon using the ideal gas law.

\[
P_{super} = P_{gas} - P_{air} \ \ \ \ (1)
\]

We can write the ideal gas equation for the gas inside the balloon:

\[
 P_{gas}V = {m_{gas}\over{M_{gas}}} R T_{gas} \ \ \ \ (2)
\]

Since the system is floating, we know the mass of the air displaced is
$m_{system}$. So we can also write the idea gas equation for the air
displaced by the balloon.

\[
 P_{air}V = {m_{system}\over{M_{air}}} R T_{air} \ \ \ \ (3)
\]

We assume the volumes are equal, so we can subsitiute one into the other.

\[
 P_{gas} = P_{air} \bigg[ {{{m_{gas}\over{M_{gas}}} R T_{gas}}\over{{m_{system}\over{M_{air}}} R T_{air}}}  \bigg]  \ \ \ \ (4)
\]

Re-arrange and cancel $R$:

\[
 P_{gas} = { P_{air} \bigg[ { {m_{gas}T_{gas}M_{air}}\over{M_{gas}T_{air}m_{system}} } \bigg]}  \ \ \ \ (5)
\]

Now we can use the definition of superpressure (1):

\[
P_{super} = P_{gas} - P_{air} \ \ \ \ (1)
\]

\[
 P_{super} = { P_{air} \bigg[ { {m_{gas}T_{gas}M_{air}}\over{M_{gas}T_{air}m_{system}} } - 1\bigg]}  \ \ \ \ (6)
\]

Substituting in our expression for $P_{air}$:

\[
 P_{air} = {{m_{system}R T_{air}}\over{M_{air}V}} \ \ \ \ (3)
\]

\[
 P_{super} = { {{m_{system}R T_{air}}\over{M_{air}V}} \bigg[ { {m_{gas}T_{gas}M_{air}}\over{M_{gas}T_{air}m_{system}} } - 1\bigg]}  \ \ \ \ (7)
\]

\[
 P_{super} = { {R\over{V}} \bigg[ { {m_{gas}}\over{M_{gas}} } T_{gas} - { {m_{system}}\over{M_{system}} } T_{air}\bigg]}  \ \ \ \ (8)
\]

We define supertemperature in the same way as superpressure:

\[
 T_{super} = T_{gas} - T_{air} \ \ \ \ (9)
\]

\[
 P_{super} = { {R\over{V}} \bigg[ \Big( {m_{gas}\over{M_{gas}}} - {m_{system}\over{M_{air}}} \Big)T_{air}   +  {{m_{gas}}\over{M_{gas}}}T_{super} \bigg]} \ \ \ \ \ \ (10)
\]

We can reasonably say the superpressure due to the temperature dominates, so

\[
 {P_{super}\over{T_{super}}} \approx   {{m_{gas}}\over{M_{gas}}}{R\over{V}} \ \ \ \ \ (11)
\]

\end{document}