\documentclass{beamer}
\usepackage[latin1]{inputenc}
\usepackage{hyperref}
\title[Small superpressure]{A quick guide to small superpressure}
\subtitle{\url{https://github.com/richardeoin/a-quick-guide}}
\author{Richard Meadows}
\institute{UKHAS Conference 2016}
\date{}
\begin{document}

\begin{frame}
  \titlepage
\end{frame}


\begin{frame}{Superpressure is.. }

  \begin{columns}
    \begin{column}{0.6\textwidth}
      \begin{itemize}
      \item Gas sealed within the envelope.
        %% if the balloon is to do anything useful, this gas will end
        %% up at a higher pressure than the surrounding air - hence
        %% the name
      \item Envelope is intended to be inelastic.
        %% that is, the envelope will stop stretching and become
        %% stable, The resut of this is that the balloon remains at a
        %% particular density-altitude.

      \end{itemize}

    \end{column}
    \begin{column}{0.4\textwidth}
      \begin{figure}[!ht]
        %% image of lally balloon
        \includegraphics[width=1\textwidth]{lally_1967_balloon.png}
        \caption{GHOST Balloon, Lally 1967}

        %% this image is from when first
      \end{figure}
    \end{column}
  \end{columns}


\end{frame}

\begin{frame}{Can Amateurs do this too?}

  \begin{itemize}
  \item Yes!
  \item See also Dan Bowen at \href{https://ukhas.org.uk/general:ukhasconference}{UKHAS 2011}.
  \end{itemize}

  \begin{columns}
    \begin{column}{0.5\textwidth}
      \begin{figure}[!ht]
        %% ubseds6
        \includegraphics[width=1\textwidth]{ubseds6_altitude_plot.png}
        \caption{UBSEDS6, 7th June 2015}
      \end{figure}

    \end{column}
    \begin{column}{0.5\textwidth}
      \begin{figure}[!ht]
        %% image of b-64
        \includegraphics[width=1\textwidth]{B-64-all.jpg}
        \caption{B-64, Leo Bodnar 2014}
      \end{figure}
    \end{column}
  \end{columns}

  % Multi-day flights with small envelopes (1-2 meters on the longest axis).
  % Leo flight -- 134 days

  %% go back and check Dan's presentation too - I haven't got time to
  %% return to everything he discussed.

\end{frame}

%% What does one look like in flight?

\begin{frame}{In Flight}

  \begin{figure}[!ht]
    %% image of ubseds20
    \centering
    \includegraphics[width=0.8\textwidth]{UBSEDL_2016-08-29T10-24-37_3.png}
    \caption{UBSEDS20 balloon at 12.5km float, 29th August 2016}
  \end{figure}

  %% lots of people here contributed to this image..

\end{frame}
\begin{frame}{Floating}

  % Floating - what does this mean?
  % calcualate density

  Float when:

  \[
    \text{Atmospheric Density} = \text{System Density} = {\frac{\Sigma{m}}{V}}
  \]

  %% we can assume that the payload has no volume, and the same for
  %% the material that makes the balloon.

  However, the balloon envelope stretches somewhat:
  % Envelope isn't perfectly inelastic

  \[
    V = V_{initial}\times\Gamma
  \]

  %% introduce gamma as ratio Vfloat / Vbuilt

  %% atmospheric density profile
  \begin{figure}[!ht]
    \centering
    \includegraphics[width=0.8\textwidth]{isa_density_profile.png}
    \caption{Density in the International Standard Atmosphere}
  \end{figure}


\end{frame}
\begin{frame}{The Origins of Superpressure}

  %% Superpressure - where does this come from?

  \begin{itemize}
  \item Free lift
    %% more mols of gas inside than displaced outside
  \item Supertemperature
    %% aka. superheat, initial studies tend to use supertemperature,
    %% so we'll stick with that. Floating greenhouse.
  \item Vertical Air Currents (Lally 1967, VI. D. p.31)
    %% less significant, < 10%
  \end{itemize}

\end{frame}

\begin{frame}{Calculating Superpressure 1}

  Ideal gas law $PV = nRT$

  \begin{columns}
    \begin{column}{0.5\textwidth}
      % gas
      \begin{figure}[!ht]
        \centering
        \includegraphics[width=0.6\textwidth]{circle_gas.png}
      \end{figure}

      \[
        P_{gas}V = {m_{gas}\over{M_{gas}}} R T_{gas}
      \]
    \end{column}
    \begin{column}{0.5\textwidth}
      % displaced air
      \begin{figure}[!ht]
        \centering
        \includegraphics[width=0.6\textwidth]{circle_air_displaced.png}
      \end{figure}

      \[
        P_{air}V = {m_{system}\over{M_{air}}} R T_{air}
      \]
      % can say this because we're floating
    \end{column}
  \end{columns}

  % now make volumes equal, and cancel R

\end{frame}

\begin{frame}{Calculating Superpressure 2}

  Definitions of Superpressure and Supertemperature:
  % aka. superheat

  \[
    P_{super} = P_{gas} - P_{air}
  \]
  \[
    T_{super} = T_{gas} - T_{air}
  \]

  Assuming volumes are equal:

  % taking the equation on the previous page, and after some algebra..
  % algebra is available as a separate document
  \[
    P_{super} = { {R\over{V}} \bigg[ \Big( {m_{gas}\over{M_{gas}}} - {m_{system}\over{M_{air}}} \Big)T_{air}   +  {{m_{gas}}\over{M_{gas}}}T_{super} \bigg]}
  \]

  % first term is due to extra gas - free lift, second due to supertemperature

  The second term dominates, so:

  \[
    {P_{super}\over{T_{super}}} \approx   {{m_{gas}}\over{M_{gas}}}{R\over{V}}
  \]

  % So superpressure and supertemperature are proportional - this is
  % well known (Lally etc.) - and we want to minimise the constant of
  % proportionality.

\end{frame}

% \item Effects of changing gamma.


\begin{frame}{Supertemperature}

  \begin{figure}[!ht]
    %% lally table
    \centering
    \includegraphics[width=0.8\textwidth]{lally_19_table_9.png}
    \caption{Lally 1967, Table 9 p.24 (edited)}
  \end{figure}

  % this gives us a useful guesstimate at the supertemperature

\end{frame}

% I noted earlier that amateur balloons aren't spherical. Instead
% they're make flat and then inflated. Bristol SEDS, Leo, Qualatex
% are all essentially this shape. It's easy to make.

\begin{frame}{Mylar Balloon Shape 1}

  % This is the "mylar balloon".
  % shape. So called because mathematicians found this shape "in the
  % wild" and named it after the object that took this shape - namely
  % party balloons made from mylar.

  \begin{figure}[!ht]
    %% mylar balloon shape
    \centering
    \includegraphics[width=0.7\textwidth]{paulsen_1994_figure_1.png}
    \caption{Paulsen 1994, Figure 1}
  \end{figure}

  \[
    \int_{0}^{a} \sqrt {1 + f'(x)^2}\ dx = r
  \]

  % When you inflate it, the radius that the 2D shape had still
  % exists. So it limits the shape

  % This is a well defined shape, can calcuate volume and so on - for
  % instance the area of this cross section is 2 a^2

\end{frame}

\begin{frame}{Mylar Balloon Shape}

  \begin{figure}[!ht]
    \centering
    \includegraphics[width=1\textwidth]{mylar_balloon_crimping_hot.png}
    \caption{Crimping means a small area the in centre is stressed. }
  \end{figure}

  %% The size of the area that's stressed is related to the
  %% elasticisty of the material, which probably is quite low at
  %% stratospheric temperatures.

  %% So this design doesn't appear to be much better than the tetroon,
  %% where stress is concentrated at the corners.

  %% But we've got a trick...

\end{frame}

\begin{frame}{The Magic of Pre-stretch}

  %% Major step in making these balloons work - attributed to whom??

  \begin{itemize}
  \item Minimise Creep and relieve manufacturing stresses (Lally 1967, VI. C. p.28)
    %% Lally knew about this
  \item Increases $\Gamma$, leading to higher float and lower superpressure.
    % our equation for density has volume on the bottom - we increase
    % volume, get less dense and go higher. Same for pressure-thermal ratio
    % Gamma ~1.7 for latest flights
  \item Re-distributes stresses around mylar balloon shape.
    %% When first built the stress is concentrated in the middle of each gore.
    %% Pre-stretching equalises the stress over a much greater proportion of the gore.

    %% Pre-stretch generally good, as long as your material
    %% mantains its properties. We haven't explored gamma > 2 regime
    %% however.

  \end{itemize}

\end{frame}

\begin{frame}{Envelope Construction}

  \begin{figure}[!ht]
    \centering
    \includegraphics[width=0.9\textwidth]{bristol_seds_balloon_1_9m.png}
    \caption{Drawing for 1.9m balloon}
  \end{figure}

\end{frame}

\begin{frame}{Envelope Construction}

  \begin{figure}[!ht]
    \centering
    \includegraphics[width=1\textwidth]{bristol_seds_balloon_1_9m_film.png}
    \caption{50$\mu$m film cross section}
  \end{figure}

  Thanks to Exploratory Ideas grant from CEOI.
  %% Paid for the lab time to take a look at this

\end{frame}

\begin{frame}{Further Work}

  \begin{itemize}
  \item Web based calcuator - like the Burst Calculator.
  \item Numerical analysis of previous flights.
  \item Guidelines for minimum free lift.
    %% drag equation
  \item Modelling and measuring supertemperature.
    %% not so easy, but do-able
  \item Model for mylar tube shape.
    %% bit of geometry
  \item Explore $\Gamma > 2$
    %% the limit of pre-stretch
  \item Measuring strain on the ground (Angell and Pack, Apr. 1960).
    %% no specilist tools needed
  \item Relationship between stress and strain.
    %% in non-linear region - okay this is hard

  \end{itemize}

\end{frame}

\begin{frame}{Further Work}

  \begin{itemize}
  \item Have fun flying round the world...
  \end{itemize}

  \begin{figure}[!ht]
    \centering
    \includegraphics[width=0.6\textwidth]{pico-pi-logo.png}
  \end{figure}

\end{frame}

\begin{frame}{Meridianal Hoop}

  \begin{figure}[!ht]
    \centering
    \includegraphics[width=1\textwidth]{mylar_balloon_meridianal_hoop.png}
    \caption{Meridianal Hoop of a Mylar Balloon }
  \end{figure}

\end{frame}


\end{document}